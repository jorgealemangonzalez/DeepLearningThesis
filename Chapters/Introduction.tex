\chapter{Motivation}
Nature has the ability of evolving in all kinds of adverse environments.
One of the consequences of this continuous evolving is us, human beings,
the maximum expression of adaptive living being.
By using our intelligence we improve year by year the chances of survival
to this chaotic world.
This kind of intelligence is composed by several cognitive abilities: learning,
forming concepts, understanding, applying logic and reasoning.

Since creation of \acf{AI} investigators from all around
the world have tried to grant machines some of this skills.
One way of letting machines exhibit intelligence is by modeling how the human brain works.

The brain is composed by billions of neurons that interact with each other to create intelligence.
This system is named \acf{ANN} and scientists have learned lots of things about how it works,
but not everything.
Mathematicians used this knowledge to develop an artificial version of this network called
\acf{ANN}, which models neural networks based on mathematical functions.

But, why it is important for us to have intelligent machines?.
Because by teaching computers to do some tasks they will help us reducing the amount of work.
We can create intelligent tools to do our work better or to make our lives easier.

In the last years an area of \ac{AI} called \acf{RL} has become really popular.
In this field algorithms learn how to solve any kind of tasks that has a numeric reward attached.
There are plenty of algorithms capables of learning abstractions of the problem to reuse its knowledge in similar ones.
Most of them are based on making decisions one after the other and looking how this choices change the environment.

The goal of this thesis is to extend an specific \ac{RL} algorithm based on \acp{ANN} to help machines approach their
intelligence to ours.

%%% Local Variables: 
%%% mode: latex
%%% TeX-master: "../report"
%%% End: 
