%DIJOUS
\newcommand{\refscreen}[1]{\hyperref[fig:montezuma-map]{screen~#1}}
\chapter{Methodology}

\section{Algorithms}
I used tensorflow...

\subsection{\acl{A3C}}
%TODO SEPARAR SECCIONES EN DISTINTAS HOJAS?

\subsection{\acl{MA3C}\label{subsec:MA3C}}

\section{Environments}
The ultimate goal of this thesis is to play Montezuma's revenge, but since it is a complex game i have developed two
simpler environments.
Both helped me to understand the pros and cons of \ac{A3C} and to develop a new version of this algorithm that uses subtasks.

\subsection{Simple States}

This game is made by a $6 \times 10$ grid of square blocks.
Their colors define the kind of object and how they will interact with the hero (an special object).
This are the different types:
\begin{itemize}
  \item \newconcept{hero}: A block that can be controlled by the actions.
  \item \newconcept{wall}: If the hero hits a wall the game ends and he obtains a reward of value -1.
  When this happens we say that the hero dies.
  \item \newconcept{checkpoint}: When the hero reach this object he obtains a reward of value 1 and enables the hidden
  checkpoint reward.
  Once the hero reaches it for the first time he will never obtain the reward again.
  \item \newconcept{hidden checkpoint}: When the hero reach this object enables the door.
  It basically forces the hero to pass and there is no reward when this happens.
  \item \newconcept{door}: If the hero reach the door having gone through the different checkpoints the game ends and
  he obtains a reward of value 1.
\end{itemize}

As you can imagine the goal of the hero is to reach the door by going through the different checkpoints without colliding
with any wall, the maximum score is 2.
The game is organized in three different phases/states.
The objective of the first one is to reach the checkpoint,
the second one the hidden checkpoint and the third one the door.
The information about the current phase is available
to the player.

The purpose of creating this game is to prove that the \ac{MA3C} algorithm (\ffref{subsec:MA3C}) has better exploration
skills than A3C (\ffref{subsec:A3C}).

\subsection{Complex States}
This game is really similar to the previous one but the dynamics are a bit different.
The hero must follow a series of steps to reach the door.
The order in which the hero must go through the objects is: checkpoint, hidden checkpoint, door, hidden checkpoint,
checkpoint, door.
We force the hero to go back on his own steps when he reaches the door.
In this game some of the rewards also change.
This are the changes respect to Simple States game, the rest remains equal:
\begin{itemize}
    \item \newconcept{checkpoint}: The first time that the hero goes through this object he obtains a reward of 1.
    Then he must follow the rest of the path described above to obtain again 1 of score (after the second time he visits the hidden checkpoint).
    \item \newconcept{hidden checkpoint}: In this game both 2 times that the hero goes through this object
    (following the path) receives 1 of score.
    The first time the checkpoint object enables the reward and the second time the door enables it.
    \item \newconcept{door}: The first time the hero reach the door it moves to the starting position of the hero and
    gives him 1 of score.
    The second time behaves as in Simple States.
\end{itemize}

In this game there are only two phases.
The first one goes from the start until the first time the hero reaches the door.
The second one finish the second time it reaches the door.
As in Simple States the hero must follow the path in order to obtain the rewards and finish the game.
The information about the current phase is also available to the player.

The purpose of creating this game is to prove that \ac{A3C} is quite bad going back on his own steps (when the image of the
game is similar) while for \ac{MA3C} is easy.
%%% Local Variables: 
%%% mode: latex
%%% TeX-master: "../report"
%%% End: 
